\documentclass[10pt,twocolumn]{article}
\usepackage{latexsym,amssymb,enumerate,amsmath,epsfig,amsthm}
\usepackage[margin=1in]{geometry}
\usepackage{setspace,color}
\usepackage{parskip}
\usepackage{graphicx}
\usepackage{subfigure}
\usepackage[english]{babel}
\usepackage[table,xcdraw]{xcolor}
\usepackage[utf8]{inputenc}
\usepackage{amsmath}
\usepackage{graphicx}
\usepackage[colorinlistoftodos]{todonotes}
\usepackage{geometry}
\usepackage{caption}
\usepackage{url}
\usepackage{array}
\usepackage[toc,page]{appendix}

\usepackage{tikz}
\usetikzlibrary{shapes,arrows}
\tikzstyle{block} = [rectangle, draw, text width=7.5em, text centered, rounded corners,node distance=4cm, minimum height=4em]
\tikzstyle{line} = [draw, -latex']

\newtheorem{eg}{Example}[section]
\newcommand{\ds}{\displaystyle}
\usepackage{hyperref}
\usepackage{xcolor}
\hypersetup{
	colorlinks,
	linkcolor={red!50!black},
	citecolor={blue!50!black},
	urlcolor={blue!80!black}
}

\begin{document}
\title{Simulación de una cola en una biblioteca}
\author{
	Amanda Cordero Lezcano, Facultad de Matemática y Computación, Universidad de La Habana\\
	Christopher Guerra Herrero, Facultad de Matemática y Computación, Universidad de La Habana\\
	Alfredo Nuño Oquendo, Facultad de Matemática y Computación, Universidad de La Habana\\
	%\thanks{MATH 4336 - Introduction to Mathematics of Image Processing, %Instructor: {\textit{Prof. Shingyu LEUNG}}, Teaching Assistant: %{\textit{Mr. Ka Wah WONG}}}
}
\markboth{Homer Lee}{SSW Application}

\twocolumn[
	\begin{@twocolumnfalse}
		\maketitle
		\begin{abstract}
			Este estudio analiza, a partir de simulaciones, el rendimiento de una biblioteca en la que se evalúan las políticas de atención SJF (Shortest Job First) y FIFO(First In First Out), y se varía el número de bibliotecarios. El objetivo es entender cómo estas políticas impactan en varios indicadores de rendimiento, tales como el tamaño de la cola, el tiempo de espera, y el número de clientes atendidos.
		\end{abstract}
		\vspace{1cm}
	\end{@twocolumnfalse}
]




% ************************************************************
% ******************* INTRODUCCION ***************************
% ************************************************************

\section{Introducción}
El manejo eficiente de las colas en las bibliotecas es crucial para garantizar una experiencia positiva tanto para los clientes como para los trabajadores. No resulta agradable o atractivo un local en el que la demanda de los clientes sea superior a la capacidad de atención por parte de los trabajadores del centro.

\subsection{Breve descripción del proyecto}
En este análisis, efectuamos simulaciones del funcionamiento de la biblioteca y nos enfocamos en examinar estadísticamente los resultados de las simulaciones.

\subsection{Objetivos y metas}
Nuestra investigación busca comprender cómo diferentes políticas de atención, como la política Shortest Job First (SJF), afectan el rendimiento de la biblioteca en términos de tiempo de espera, tamaño de la cola, número de usuarios atendidos y que se retiran de la biblioteca sin ser atendidos. Mediante el análisis estadístico de los datos recogidos durante la simulación, buscamos identificar patrones y tendencias que puedan guiar la implementación de estrategias más efectivas para minimizar los tiempos de espera y maximizar la satisfacción de los clientes.

\subsection{El sistema específico a simular y las variables de interés}
El sistema a simular será el funcionamiento de la biblioteca en un período de tiempo(8 horas). De los resultados de las simulaciones, además de su relación con las variables del problema nos interesan los siguientes datos:
\begin{itemize}
	\item Total de clientes atendidos
	\item Tiempo de espera máximo y promedio en la cola
	\item Tamaño máximo y promedio de la cola
	\item Total de clientes que se retiran sin ser atendidos
\end{itemize}

\subsection{Variables que describen el problema}
El problema es descrito por las siguientes variables:
\begin{itemize}
	\item Distribución de llegada de los clientes
	\item Distribución del tiempo que tardan en atender los bibliotecarios
	\item Tiempo promedio de espera y atención
	\item Número de bibliotecas involucradas en la simulación
	\item Indicador de si se usó la política SJF(en caso contrario se usa FIFO)
	\item Tiempo que esperan los clientes en la cola antes de retirarse
\end{itemize}






% ************************************************************
% ******************* CAPITULO 2 ***************************
% ************************************************************

\section{Detalles de Implementación}

\subsection{Pasos seguidos para la implementación}
% Metodología, Algoritmo de simulación, Configuración del entorno, Validación del modelo
La implementación consta de dos partes. Primeramente se simulan varias jornadas del funcionamiento de la biblioteca y la interacción entre las entidades que participan. Luego como resultado de las simulaciones se tienen almacenados un conjunto de datos, los cuales son utilizados en un análisis estadístico.

\subsection{Descripción de la Simulación}
La simulación modela el funcionamiento de una biblioteca en la que los clientes llegan con una distribución de Poisson al mostrador, con una media de 10 (representa la cantidad media de clientes por hora) y esperan ser atendidos por bibliotecarios. La cantidad de bibliotecarios varía entre 1, 2 y 3 en las diferentes simulaciones. La política de atención puede variar entre FIFO y SJF entre una simulación y otra. Los servicios requeridos por los clientes se clasifican en rápidos, normales y lentos. Los bibliotecarios atienden a cada cliente de servicio rápido con una distribución exponencial de media 2. A los clientes con servicios normales y lentos se le calcula su tiempo de atención como a los rápidos pero una vez determinado este se le suma 3 y 6 minutos respectivamente.
Se registran varias métricas clave durante la ejecución de la simulación, como son ciertos datos relacionados con la cola de espera de los clientes (tamaño y tiempo de espera), la cantidad de clientes atendidos y la cantidad que se retiran sin ser atendidos. Estas se analizan para evaluar el desempeño bajo las diferentes políticas.

{\color{gray}
\subsection{Métodos de Análisis Estadístico}
Se realizaron análisis descriptivos de los datos tales como: la prueba de T-Student, ANOVA y Spearman para entender la distribución de las diferentes métricas. Una de los aspectos a analizar fue comparar los resultados entre simulaciones que usaron la política SJF y las que no la usaron para evaluar el impacto de esta política en el rendimiento de la biblioteca.
}



% ************************************************************
% ******************* CAPITULO 3 ***************************
% ************************************************************

\section{Resultados y Experimentos}
\subsection{Hallazgos de la simulación}
A lo largo de la simulación monitoreamos varias métricas clave como son: políticas de manejo de colas y cantidad de estudiantes trabajando en la biblioteca. Como resultado de ello ahora contamos con un dataset que contiene datos de observaciones en distintas jornadas de trabajo. La simulación exploró además distintos tiempos antes de que un cliente se retirase sin ser atendido para así valorar distintos escenarios del problema en cuestión.

{\color{gray}
\subsection{Interpretación de los resultados}
Los resultados de la simulación ofrecen una visión valiosa sobre el rendimiento de la biblioteca bajo diferentes condiciones. Analizando las estadísticas descriptivas presentadas en la Tabla \ref{tab:descriptive_stats}, podemos inferir varios puntos clave:

\begin{itemize}
	\item El tamaño máximo y promedio de la cola (\texttt{max\_size\_queue}, \texttt{ave\_size\_queue}) reflejan la carga operativa de la biblioteca. Un tamaño de cola alto indica que hay una demanda alta pero también que la política de atención podría necesitar ajustes para mejorar la eficiencia.
	\item Los tiempos de espera y atención (\texttt{ave\_wait\_time}, \texttt{ave\_attention\_time}) son cruciales para la satisfacción del cliente. Los tiempos de espera largos pueden llevar a una disminución en la satisfacción del cliente, mientras que los tiempos de atención demasiado cortos pueden indicar una falta de recursos humanos adecuada.
	\item El total de clientes atendidos (\texttt{total\_clients\_attended}) es un indicador directo de la capacidad de la biblioteca para manejar la demanda. Un número alto de clientes atendidos sugiere una buena eficiencia operativa.
	\item El número de bibliotecarios involucradas (\texttt{libraries\_count}) sugiere que la expansión o la redistribución de recursos podría ser una estrategia viable para manejar picos de demanda.
	\item La aplicación de la política SJF (\texttt{sjf}) parece influir significativamente en el rendimiento de la biblioteca.
\end{itemize}

Es importante considerar que la variabilidad en los resultados (reflejada por la desviación estándar en \textbf{Tabla 1}) indica que existen factores externos o condiciones específicas de la simulación que pueden influir en dichos resultados.

\subsection{Hipótesis extraídas de los resultados}
Como puede observarse en los BoxPLot parece haber cierta relación entre la aplicación de la política SJF y la mejora la media de tiempo de espera de los clientes (cita), esta es una de las hipótesis que se pueden extraer de los datos.
Otra hipótesis sería la posibilidad de que el añadir un quinto bibliotecario no afecte en sentido general a las métricas analizadas.

\subsection{Experimentos realizados para validar las hipótesis}
Para el análisis de la primera hipótesis se realizo un análisis de la correlación de las variables de salida con respecto a la aplicación de la política SJF (adjuntar foto). Como puede observarse la aplicación de esta política no tiene correlaciones significativas con ninguna de las variables de salida, lo cual entra en contradicción con lo que se observa en el BoxPlot. \\
Para comprobar la relación entre las variables SJF y Mean Wait Time se utilizó una prueba  Mann-Whitney U la cual es una alternativa a la prueba t de Student cuando no se cumplen los supuestos de normalidad. La misma arrojó los resultados siguientes:\\
MannwhitneyuResult(statistic=1248.0, pvalue=0.8624062624558004) (codigo)\\
El valor p es 0.8624, que es mucho mayor que 0.05. Esto indica que no hay suficiente evidencia para rechazar la hipótesis nula. En otras palabras, no hay una diferencia significativa en el tiempo medio de espera (Mean Wait Time) entre las configuraciones de "SJF=True" y "SJF=False".\\
\\
El BoxPlot indica que al añadir un quinto bibliotecario los datos tienen un comportamiento similar al que tenian con 4. Esto significa que solo representaría un costo adicional a la biblioteca y objetivamente no es necesario. Para validar esta hipótesis se realizó una prueba t de Student (ya que se pudo verificar que se cumplian los supuestos de la misma) para comprobar si existe una diferencia significativa en la cantidad de clientes atendidos en las configuraciones de 4 y 5 bibliotecarios (también se realiza un análisis similar para la variable tiempo medio de espera)\\
Los resultados son:\\
(Cantidad de clientes atendidos) T-test Result: TtestResult(statistic=1.56677385807371, pvalue=0.12417347029059797, df=45.0)\\
(Tiempo medio de espera) T-test Result: TtestResult(statistic=-0.748423343123664, pvalue=0.45809885435813613, df=45.0)\\
En ambos casos el pvalue es mayor que 0.05 por lo cual podemos concluir que no existe diferencia significativa en estas metricas en los grupos de datos de 4 y 5 bibliotecarios, por lo cual podemos concluir que la adición de un quinto bibliotecario no mejora significativamente el servicio en la biblioteca y por lo tanto no es necesario.




\subsection{Análisis de Parada de la Simulación}
La elección del momento exacto para detener una simulación es crítica, ya que puede influir significativamente en los resultados obtenidos. En nuestro estudio, adoptamos un enfoque sistemático para determinar el criterio de parada, basándonos en consideraciones tanto prácticas como metodológicas.

\subsubsection{Criterios de Parada}
Los criterios de parada se establecieron con el objetivo de alcanzar un equilibrio entre la precisión de los resultados y la eficiencia en el tiempo de ejecución de la simulación. Entre los criterios considerados se encuentran:

\begin{itemize}
	\item \textbf{Tiempo de Simulación}: Se estableció un límite de tiempo para la ejecución de la simulación, asegurando que se obtengan resultados en un marco de tiempo manejable.
	\item \textbf{Número Máximo de Iteraciones}: Se definió un número máximo de iteraciones para evitar la ejecución indefinida de la simulación, lo cual podría llevar a resultados menos representativos.
	\item \textbf{Estabilización de Métricas Clave}: Observamos la estabilización de las métricas clave (por ejemplo, el tamaño de la cola, el tiempo de espera medio) a lo largo de múltiples intervalos de tiempo, lo que indicaba que la simulación había alcanzado un estado estacionario.
\end{itemize}

\subsubsection{Razonamiento detrás de la Elección de los Criterios}
La selección de estos criterios se fundamenta en la necesidad de obtener resultados confiables y reproducibles. El límite de tiempo y el número máximo de iteraciones sirven para controlar el costo computacional, evitando simulaciones excesivamente largas que podrían no aportar información adicional útil. Por otro lado, la estabilización de las métricas clave es crucial para asegurar que los resultados sean representativos de un estado estacionario del sistema, evitando así el sesgo hacia fases iniciales de la simulación.

\subsubsection{Implicaciones para el Análisis}
La elección cuidadosa de los criterios de parada tiene implicaciones directas para el análisis posterior de los resultados. Al detener la simulación en un punto donde las métricas clave han alcanzado un estado estacionario, nos aseguramos de que los hallazgos sean robustos y generalizables. Además, al considerar el tiempo de simulación y el número máximo de iteraciones, mantenemos un equilibrio entre la profundidad del análisis y la viabilidad operativa de la simulación.
}

% ************************************************************
% ******************* CAPITULO 4 ***************************
% ************************************************************


\section{Modelo Matemático}

\subsection{Descripción del modelo}

El sistema de cuatro bilbiotecarios con política FIFO se basa en un modelo matemático de un sistema de colas con cuatro servidores en paralelo. En este sistema, los clientes llegan siguiendo un proceso de Poisson con una tasa de llegada $\lambda$, y los tiempos de servicio son variables aleatorias exponenciales con tasa $\mu$. Los servidores operan en paralelo, lo que significa que pueden atender a varios clientes simultáneamente.

Este tipo de sistema es conocido como un modelo $M/M/4$ en la teoría de colas, donde:

\begin{itemize}
	\item $M$ denota que los tiempos entre llegadas siguen una distribución de Poisson.
	\item $M$ denota que los tiempos de servicio siguen una distribución exponencial.
	\item 4 indica que hay cuatro servidores en paralelo.
\end{itemize}

El objetivo del modelo es analizar el tiempo promedio de espera en la cola, la probabilidad de que un cliente tenga que esperar para ser atendido, y la utilización de los servidores.

\subsection{Supuestos y restricciones}

El modelo $M/M/4$,se basa en los siguientes supuestos y restricciones:

\begin{enumerate}
	\item \textbf{Proceso de Llegada}: Los clientes llegan de acuerdo a un proceso de Poisson con tasa $\lambda$. Esto implica que las llegadas son independientes y el tiempo entre llegadas sigue una distribución exponencial.
	\item \textbf{Tiempo de Servicio}: Los tiempos de servicio de los servidores son variables aleatorias exponenciales con tasa $\mu$. Esto asegura que la duración del servicio es independiente del tiempo transcurrido.
	\item \textbf{Capacidad de la Cola}: Se asume que la cola tiene capacidad infinita, es decir, no hay límite en el número de clientes que pueden esperar en la cola.
	\item \textbf{Disciplina de la Cola}: Los clientes son atendidos en el orden de llegada (FIFO - First In, First Out).
	\item \textbf{Independencia}: Se asume que los tiempos de llegada y servicio son independientes entre sí y no varían con el tiempo.
	\item \textbf{Servidores Idénticos}: Los dos servidores son idénticos en términos de capacidad y eficiencia, es decir, ambos tienen la misma tasa de servicio $\mu$.
\end{enumerate}

\subsection{Comparación de los resultados obtenidos}

Para evaluar la efectividad del modelo, se comparan los resultados obtenidos de las simulaciones con los resultados teóricos esperados del modelo $M/M/4$. Los principales indicadores a considerar incluyen:

\begin{itemize}
	\item {Tiempo Promedio de Espera}
	\item {Probabilidad de Espera}
	\item {Utilización de los Servidores}
\end{itemize}

Los resultados de la simulación deben mostrar una buena correspondencia con los valores teóricos, validando así el modelo utilizado. Cualquier discrepancia significativa puede indicar la necesidad de revisar los supuestos del modelo o la implementación de la simulación.

\color{gray}
\section{Conclusiones}
En resumen, este estudio proporciona una visión detallada del rendimiento de una biblioteca bajo diferentes políticas de atención. Los resultados indican que la política SJF puede tener un impacto significativo en la eficiencia operativa de la biblioteca, reduciendo potencialmente los tiempos de espera y mejorando la experiencia del cliente. Futuros investigaciones podrían explorar más a fondo las condiciones óptimas para implementar esta política y otras estrategias de gestión de colas.



\end{document}
